% %%%%%%%%%%%%%%%%%%%%%%%%%%%%%%%%%%%%%%%%%%%%%%%%%%%%%%%%%%%%%%%%%%%%%%%%%%% 
%
% Map table elements to extreme table (xtable) macros
%
% %%%%%%%%%%%%%%%%%%%%%%%%%%%%%%%%%%%%%%%%%%%%%%%%%%%%%%%%%%%%%%%%%%%%%%%%%%% 

\startxmlsetups xml:table
  \blank
  \startembeddedxtable
    \xmlflush{#1}
  \stopembeddedxtable
  \blank
\stopxmlsetups

\startxmlsetups xml:thead
  \startxtablebody[head]
    \xmlflush{#1}
  \stopxtablebody
\stopxmlsetups

\startxmlsetups xml:tbody
  \startxtablebody[body]
    \xmlflush{#1}
\stopxmlsetups

\startxmlsetups xml:tfoot
  \startxtablebody[foot]
    \xmlflush{#1}
  \stopxtablebody
\stopxmlsetups

\startxmlsetups xml:tr
  \startxrow
    \xmlflush{#1}
  \stopxrow
\stopxmlsetups

% When the XHTML table lacks a tfoot element, treat the last table row
% in the body as part of the footer, which may be styled independently.
\startxmlsetups xml:tr:last
  \stopxtablebody
  \startxtablebody[foot]
  \startxrow\xmlflush{#1}\stopxrow
  \stopxtablebody
\stopxmlsetups

\startxmlsetups xml:th
  \startxcell
    \bold{\xmlflush{#1}}
  \stopxcell
\stopxmlsetups

\startxmlsetups xml:td
  \startxcell
    \xmlflush{#1}
  \stopxcell
\stopxmlsetups

